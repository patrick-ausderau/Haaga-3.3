\section*{3.3 Communities and Networks - 2\textsuperscript{nd} Study Circle}

Hytönen et al. \cite{hytonen2014does} describe professions as dynamic. Established professions evolve and new ones emerge based on societal challenges and interests, new technologies and its evolution and changes in legislations. However, with rapid changes, the pathways for developing expertise might not exist and be left out to the worker to set. This require that s/he can collaborate effectively with people facing similar challenges as well as being able to move beyond one's own discipline.

Hytönen et al. \cite{hytonen2014does} study the development of networks within a group of students during a Academic Apprenticeship Education\footnote{a model initiated in Finland in 2009 for the education in different fields of academic professional activity, funded by the Ministry of Education and Culture. The model is aimed at professionals that already work in expert tasks, which are
often multi-scientific and rapidly growing, but face a need to update and expand their expertise.} course about energy efficiency. This field is for them a good example of a domain that require the creation of versatile network relations which will enable the access to resources, information and
support from people with different expertise and competences. So they can share knowledge and find new solutions to be used in their own domain.

Hytönen et al. \cite{hytonen2014does} note that to efficiently use such network, the participants have to be aware of each other's knowledge and expertise (meta-knowledge) as well as recognize other's and self's limits. By citing Nooteboom (2004), they argue that learning through such network could append only when participants possess varied know-how while simultaneously sufficient similarities to engage in constructive dialogue.

Hytönen et al. \cite{hytonen2014does} describe Academic Apprenticeship Education as informal. The study will mix workplace learning with contact days organized by universities which contain lectures, workshops, group work and online study. With their energy efficiency study which is a new and emerging domain, they note that there are rarely other experts of the same field in the same workplace. Thus, providing a forum for sharing of knowledge and creating new professional connections is important.

Hytönen et al. \cite{hytonen2014does} conclude that a successful collaboration and network building is not possible without planning and deliberate and sustained efforts. they see as essential to examine whether the Academic Apprenticeship Education has long-term networking impacts. They end with a question: \begin{quote}
how can we better integrate the forums of learning institutions with
learning at work?
\end{quote}