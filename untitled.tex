\section*{3.3 Communities and Networks - 2\textsuperscript{nd} Study Circle}

Hytönen et al. \cite{hytonen2014does} describe professions as dynamic. Established professions evolve and new ones emerge based on societal challenges and interests, new technologies and its evolution and changes in legislations. However, with rapid changes, the pathways for developing expertise might not exist and be left out to the worker to set. This require that s/he can collaborate effectively with people facing similar challenges as well as being able to move beyond one's own discipline.

Hytönen et al. \cite{hytonen2014does} study the development of networks within a group of students during a Academic Apprenticeship Education\footnote{a model initiated in Finland in 2009 for the education in different fields of academic professional activity} course about energy efficiency. This field is for them a good example of a domain that require the creation of versatile network relations which will enable the access to resources, information and
support from people with different expertise and competences. So they can share knowledge and find new solutions to be used in their own domain.

